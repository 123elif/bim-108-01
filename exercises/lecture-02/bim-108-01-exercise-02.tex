\documentclass{beamer}
\usetheme{Boadilla}
\usecolortheme{sidebartab}
\beamertemplatenavigationsymbolsempty
\setbeamertemplate{footline}[frame number]
\usepackage{hyperref} 
\usepackage{graphicx}
\usepackage{color}
\usepackage{booktabs}
\usepackage{listings}
\usepackage[utf8]{inputenc}

\definecolor{gray}{rgb}{0.4,0.4,0.4}
\definecolor{darkblue}{rgb}{0.0,0.0,0.6}
\definecolor{cyan}{rgb}{0.0,0.6,0.6}

\lstset{
	basicstyle=\ttfamily,
	columns=fullflexible,
	showstringspaces=false,
	commentstyle=\color{gray}\upshape
}

\lstdefinelanguage{XML}
{
	morestring=[b]",
	morestring=[s]{>}{<},
	morecomment=[s]{<?}{?>},
	stringstyle=\color{black},
	identifierstyle=\color{darkblue},
	keywordstyle=\color{cyan},
	morekeywords={xmlns,version,type}% list your attributes here
}

\begin{document}

\begin{frame}[fragile]
  
  \small
  \begin{lstlisting}
Auf Desktop in Git Bash
  mkdir bim-108-01-solutions/lecture-02
  cd bim-108-01/
  git pull
  cp notebooks/lecture-02/* ../bim-108-01-solutions/lecture-02/

Anaconda Navigator 
oder Anaconda Prompt (und auf Laufwerk Z wechseln)

Uebung durchfuehren ...

Zum Schluss, auf Desktop in Git Bash
  cd bim-108-01-solutions/
  git add lecture-02
  git commit -a -m "solution to exercise 02"
  git push
  \end{lstlisting}
  
\end{frame}

\end{document}