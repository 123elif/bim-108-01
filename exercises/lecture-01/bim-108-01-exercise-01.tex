\documentclass{beamer}
\usetheme{Boadilla}
\usecolortheme{sidebartab}
\beamertemplatenavigationsymbolsempty
\setbeamertemplate{footline}[frame number]
\usepackage{hyperref} 
\usepackage{graphicx}
\usepackage{color}
\usepackage{booktabs}
\usepackage{listings}
\usepackage[utf8]{inputenc}

\definecolor{gray}{rgb}{0.4,0.4,0.4}
\definecolor{darkblue}{rgb}{0.0,0.0,0.6}
\definecolor{cyan}{rgb}{0.0,0.6,0.6}

\lstset{
	basicstyle=\ttfamily,
	columns=fullflexible,
	showstringspaces=false,
	commentstyle=\color{gray}\upshape
}

\lstdefinelanguage{XML}
{
	morestring=[b]",
	morestring=[s]{>}{<},
	morecomment=[s]{<?}{?>},
	stringstyle=\color{black},
	identifierstyle=\color{darkblue},
	keywordstyle=\color{cyan},
	morekeywords={xmlns,version,type}% list your attributes here
}

\title{Übungen}
\subtitle{Grundlagen XML und RDF}
\author{Markus Stocker}
\date{5. März 2018}

\begin{document}

\maketitle

\begin{frame}{Übungen zur Vorlesung}
  
  \begin{itemize}
  	\item Jede Vorlesung wird mit Übungen begleitet
  	\item Übungen werden als Jupyter Notebooks durchgeführt
  	\item Webbasierte interaktive Programmierung
  	\item Können auch zuhause durchgeführt werden
	\item Einreichung auf GitHub
	\item Jeweils vor der nächsten Vorlesung
  \end{itemize}
  
\end{frame}

\begin{frame}{Voraussetzungen}
	
	\begin{itemize}
		\item Anaconda: \url{https://www.anaconda.com/download/}
		\item Git Client: \url{https://git-scm.com/downloads}
	\end{itemize}
	
\end{frame}

\begin{frame}[fragile]{Anleitung}
	
	\begin{itemize}
		\item Übungen stehen auf GitHub bereit
		\item Diese von dort auf die lokale Umgebung kopieren
		\item Mittels Jupyter die Übung lokal ausführen
		\item Lösung auf eigenes GitHub Repository stellen
		\item Mehr Information hier:
	\end{itemize}
	
	\vspace{0.5cm}
	\Large\centering
  \url{https://github.com/markusstocker/bim-108-01/}
	
\end{frame}

\end{document}