\documentclass{beamer}
\usetheme{Boadilla}
\usecolortheme{sidebartab}
\beamertemplatenavigationsymbolsempty
\setbeamertemplate{footline}[frame number]
\usepackage{hyperref} 
\usepackage{graphicx}
\usepackage{color}
\usepackage{booktabs}
\usepackage{listings}
\usepackage{soul}
\usepackage{tikz}
\usepackage[utf8]{inputenc}
\usepackage{CJKutf8}

\definecolor{gray}{rgb}{0.4,0.4,0.4}
\definecolor{darkblue}{rgb}{0.0,0.0,0.6}
\definecolor{cyan}{rgb}{0.0,0.6,0.6}

\lstset{
	basicstyle=\ttfamily,
	columns=fullflexible,
	showstringspaces=false,
	commentstyle=\color{gray}\upshape
}

\lstdefinelanguage{XML}
{
	morestring=[b]",
	morestring=[s]{>}{<},
	morecomment=[s]{<?}{?>},
	stringstyle=\color{black},
	identifierstyle=\color{darkblue},
	keywordstyle=\color{cyan},
	morekeywords={xmlns,version,type}% list your attributes here
}

\makeatletter
\newcommand\SoulColor{%
	\let\set@color\beamerorig@set@color
	\let\reset@color\beamerorig@reset@color}
\makeatother

\lstset{language=XML}

\title{Schema: Document Type Definition (DTD)}
\author{Markus Stocker}
\date{9. April 2018}

\begin{document}

\maketitle

\begin{frame}{Rekapitulation}
	
	\begin{itemize}
		\item XPath, erklären Sie
		\begin{itemize}
			\item Schritt und Lokalisierungspfad: Was bedeutet \texttt{A::KT[P]*}
			\item Welche Lokalisierungspfadarten gibt es?
			\item Was sind Prädikate?
			\item Was erlauben Funktionen?
		\end{itemize}
		\item Hatten wir bereits: Was bedeutet Wohlgeformtheit?
	\end{itemize}
	
\end{frame}

\begin{frame}{Übersicht}
	
	\begin{itemize}
		\item Wozu ein Schema?
		\item Gültig: Nicht nur Wohlgeformt
		\item Die Document Type Definition (DTD)
		\item DTD Konstrukte und Beispiele
	\end{itemize}
	
\end{frame}

\begin{frame}{Schema: Warum brauchen wir sowas?}
	
	\begin{itemize}
		\item Die NASA, ESA, und JAXA haben Daten über das Sonnensystem
		\item Man möchte nun die jeweiligen Daten austauschen und integrieren
		\item Die jeweiligen Systeme stellen Daten als XML zur verfügung
		\item Allerdings verwenden diese unterschiedliche \emph{tags}
		\item Zum Beispiel \texttt{<planet/>}, \texttt{<Planet/>}, \texttt{<\begin{CJK}{UTF8}{min}惑星\end{CJK}/>}
		\item Zudem sind die XML Dokumente unterschiedlich strukturiert
		\item Solche Unterschiede erschweren den Datenaustausch
		\item Die Varianten in Software zu berücksichtigen ist mühsam
		\item Nicht nur mühsam, letztlich unmöglich da beliebig viele
		\item Was tun? 
		\item Die drei Institutionen könnten sich auf ein Schema einigen
	\end{itemize}
	
\end{frame}

\begin{frame}{Schema: Warum brauchen wir sowas?}
	
	\begin{itemize}
		\item Hier in diesem Raum legen wir uns auf die deutsche Sprache fest
		\item Würde ich die Vorlesung auf italienisch halten wäre das eher hinderlich
		\item Bei XML, der verwendeten Terme und deren Bedeutung ist es ähnlich
		\item Mit einem Schema legen sich Parteien auf einen \emph{Standart} fest
		\item Dies soll die Interoperabilität der jeweiligen Systeme ermöglichen
		\item Fähigkeit eines Systems mit anderen Systemen zusammenzuarbeiten
	\end{itemize}
	
\end{frame}

\begin{frame}{Wohlgeformtheit und Gültigkeit}
	
	\begin{itemize}
		\item Ein XML Dokument mit korrekter Syntax ist wohlgeformt
		\item Validiert ein wohlgeformtes Dokument einem Schema ist es gültig
		\item Ein gültiges Dokument ist immer auch wohlgeformt (\emph{well-formed})
		\item Ein wohlgeformtes Dokument ist nicht zwingend gültig (\emph{valid})
	\end{itemize}
	
\end{frame}

\begin{frame}{Document Type Definition (DTD)}
	
	\begin{itemize}
		\item Eine Grammatik zur Spezifikation von XML Dokumente
		\item Beschreibung gültiger Elemente und Attribute
		\item Eine DTD definiert somit auch die XML Struktur
		\item Parteien können sich auf eine gemeinsame DTD einigen
		\item Systeme werden dann entsprechend der DTD entwickelt
		\item Systeme können XML Dokumente der DTD entgegen validieren
		\item Erhaltene aber auch gelieferte XML Dokumente
		\item Validiert ein XML Dokument ist dieses ...
	\end{itemize}
	
\end{frame}

\begin{frame}[fragile]{Beispiel: XML Dokument und Entsprechende DTD}
	
	\lstset{language=XML}
	\begin{lstlisting}	
  <planets>
    <planet>
      <name>Earth</name>
    </planet>
  </planets>		
	\end{lstlisting}
	
	\vspace{1cm}
	  	
	\begin{lstlisting}	
  <!ELEMENT planets (planet)>
  <!ELEMENT planet (name)>
  <!ELEMENT name (#PCDATA)>
	\end{lstlisting}
	
\end{frame}

\begin{frame}[fragile]{Einbindung einer DTD: Im Dokument}
	
	\lstset{language=XML}
	\begin{lstlisting}	
  <?xml version="1.0"?>
  <!DOCTYPE planets [
  <!ELEMENT planets (planet)>
  <!ELEMENT planet (name)>
  <!ELEMENT name (#PCDATA)>
  ]>
  <planets>
    <planet>
      <name>Earth</name>
    </planet>
  </planets>		
	\end{lstlisting}

\end{frame}

\begin{frame}[fragile]{Einbindung einer DTD: Extern}
	
	\lstset{language=XML}
	\begin{lstlisting}	
  <?xml version="1.0"?>
  <!DOCTYPE planets SYSTEM "planets.dtd">
  <planets>
    <planet>
      <name>Earth</name>
    </planet>
  </planets>		
	\end{lstlisting}
	
  \vspace{1cm}

  Inhalt der \texttt{planets.dtd} Datei
	
	\begin{lstlisting}
  <!ELEMENT planets (planet)>
  <!ELEMENT planet (name)>
  <!ELEMENT name (#PCDATA)>
	\end{lstlisting}

\end{frame}

\begin{frame}{Konstrukte}
	
	\begin{itemize}
		\item XML besteht aus den folgenden Konstrukte
		\begin{itemize}
			\item Elemente
			\item Attribute
			\item Entitäten
			\item PCDATA (\emph{parsed character data})
			\item CDATA (\emph{[unparsed] character data})
		\end{itemize}
		\item XML Dokumente verwenden diese Konstrukte
		\item Wie wir gesehen haben, allerdings sehr unterschiedlich
		\item DTD ermöglicht Deklaration spezifischer Verwendungen
		\item Deklaration gültiger Verwendung der Konstrukte
		\item Es ist eine Grammatik zur Spezifikation von XML Dokumente
	\end{itemize}
	
\end{frame}

\begin{frame}[fragile]{Elemente Deklarieren}
	
	\begin{itemize}
		\item Elemente werden mit \texttt{ELEMENT} deklariert
		\item Die folgenden beiden Syntaxen sind erlaubt
	\end{itemize}
	
	\lstset{language=XML}
	\begin{lstlisting}
  <!ELEMENT name inhalt>
	\end{lstlisting}
	
	\begin{itemize}
		\item Der \texttt{name} des elements
		\item Und \texttt{inhalt} einer von
		\begin{itemize}
			\item \texttt{EMPTY} für das Element ohne Inhalt
			\item \texttt{ANY} für beliebige Inhalte
			\item Gemischte Inhalte, \texttt{PCDATA} und Kindelemente
			\item Kindelemente
		\end{itemize}
	\end{itemize}
	
	\begin{lstlisting}
  <!ELEMENT planets EMPTY>
  <!ELEMENT planet (name, radius)>
  <!ELEMENT name (#PCDATA)>
	\end{lstlisting}
		
\end{frame}

\begin{frame}[fragile]{Element Inhaltsmodelle}
	
	\begin{itemize}
		\item Sequenz \texttt{(A, B)}: \texttt{A} und \texttt{B} müssen in dieser Sequenz auftreten
		\item Alternative \texttt{(A | B)}: Entweder \texttt{A} oder \texttt{B} müssen auftreten
		\item Wiederholungen
		\begin{itemize}
			\item \texttt{*}: Null oder mehrmals
			\item \texttt{+}: Ein oder mehrmals
			\item \texttt{?}: Null oder einmal
		\end{itemize}
	\end{itemize}
	
	\begin{lstlisting}
  <!ELEMENT planets (planet+)>
  <!ELEMENT planet (name, radius?)>
  <!ELEMENT planet (#PCDATA | name)*>
	\end{lstlisting}
	
\end{frame}

\begin{frame}[fragile]{Attribute Deklarieren}
	
	\begin{itemize}
		\item Attribute werden mit \texttt{ATTLIST} deklariert
		\item Entsprechend der folgenden Syntax
	\end{itemize}
	
	\lstset{language=XML}
	\begin{lstlisting}
  <!ATTLIST element attribut typ wert>
	\end{lstlisting}
	
	\begin{itemize}
		\item Der \texttt{element} Name
		\item Der \texttt{attribut} Name
		\item Der \texttt{typ} des Attributs
		\item Der \texttt{wert} des Attributs
	\end{itemize}
	
	\begin{lstlisting}
  <!ATTLIST name radius CDATA "0">
	\end{lstlisting}
	
\end{frame}

\begin{frame}[fragile]{Attribut Typen (ausgewählte)}
	
	\centering
	\begin{tabular}{l|l}
	Typ & Beschreibung \\
	\hline
	\texttt{CDATA} & Wert ist Zeichenfolge \\
	\texttt{(w1|w2|...)} & Wert aus enumerierter Liste \\
	\texttt{ID} & Wert ist eindeutige ID \\
	\texttt{IDREF} & Wert ist ID eines anderen Elements \\
	\texttt{NMTOKEN} & Wert ist ein gültiger XML Name \\
	\texttt{ENTITY} & Wert ist eine Entität \\
	\hline
	\end{tabular}
	
\end{frame}

\begin{frame}[fragile]{Attribut Werte}
	
	\centering
	\begin{tabular}{l|l}
		Wert & Beschreibung \\
		\hline
		\texttt{value} & Der Standartwert des Attributs \\
		\texttt{\#REQUIRED} & Das Attribut ist zwingend \\
		\texttt{\#IMPLIED} & Das Attribut ist optional \\
		\texttt{\#FIXED value} & Das Attribut ist optional, Wert ist festgelegt \\
		\hline
	\end{tabular}
	
\end{frame}

\begin{frame}[fragile]{Entitäten Deklarieren}
	
	\begin{itemize}
		\item Entitäten werden mit \texttt{ENTITY} deklariert
		\item Entsprechend der folgenden Syntax
	\end{itemize}
	
	\lstset{language=XML}
	\begin{lstlisting}
  <!ENTITY name "wert">
	\end{lstlisting}
	
	\begin{itemize}
		\item Der \texttt{name} der Entität
		\item Der \texttt{wert} der Entität
	\end{itemize}
	
	\begin{lstlisting}
  <!DOCTYPE planet [
    <!ENTITY earth "Planet Earth">
  ]>
  <planet>
    <name>&earth;</name>
  </planet>
	\end{lstlisting}
	
\end{frame}

\begin{frame}[fragile]{Namensräume}
	
	\begin{itemize}
		\item DTD unterstüzt Namensräume nicht
		\item Man kann diese allerdings als Attribute deklarieren
	\end{itemize}
	
	\lstset{language=XML}
	
	\begin{lstlisting}
  <!ELEMENT nasa:planet EMPTY>
  <!ATTLIST nasa:planet xmlns:nasa 
             CDATA #FIXED "http://nasa.org">
	
  <nasa:planet xmlns:nasa="http://nasa.org"/>
	\end{lstlisting}
	
\end{frame}

\begin{frame}{Zusammenfassung}
	
	\begin{itemize}
		\item Schemas ermöglichen und erhöhen Interoperabilität
		\item Fähigkeit verschiedener Systeme zusammenzuarbeiten
		\item Nötig ist eine Sprache zur Spezifikation von XML Dokumente
		\item DTD ist eine solche Sprache
		\item Mit DTD kann man XML Dokumente validieren
		\item Validiert ein XML Dokument ist dieses gültig
		\item Ein gültiges XML Dokument ist immer auch wohlgeformt
	\end{itemize}
	
\end{frame}

\end{document}