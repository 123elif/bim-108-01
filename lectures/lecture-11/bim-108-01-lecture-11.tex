\documentclass{beamer}
\usetheme{Boadilla}
\usecolortheme{sidebartab}
\beamertemplatenavigationsymbolsempty
\setbeamertemplate{footline}[frame number]
\usepackage{hyperref} 
\usepackage{graphicx}
\usepackage{color}
\usepackage{booktabs}
\usepackage{listings}
\usepackage{soul}
\usepackage{tikz}
\usepackage[utf8]{inputenc}
\usepackage{CJKutf8}
\usetikzlibrary{shapes.geometric}

\definecolor{gray}{rgb}{0.4,0.4,0.4}
\definecolor{darkblue}{rgb}{0.0,0.0,0.6}
\definecolor{cyan}{rgb}{0.0,0.6,0.6}
\definecolor{lightgray}{gray}{0.8}

\lstset{
	basicstyle=\ttfamily,
	columns=fullflexible,
	showstringspaces=false,
	commentstyle=\color{gray}\upshape
}

\tikzset{node style/.style={
		draw=blue,
		thick,
		fill=blue!70,
		text=white,
		ellipse,
		minimum width=2cm,
		minimum height=0.75cm,
		font=\small,
		outer sep=3pt,
	},
	blank style/.style={
		draw=black,
		thick,
		fill=white,
		text=white,
		ellipse,
		minimum width=2cm,
		minimum height=0.75cm,
		font=\small,
		outer sep=3pt,
	},
	literal style/.style={
		draw=red,
		thick,
		fill=red!70,
		text=white,
		rectangle,
		minimum width=2cm,
		minimum height=0.75cm,
		font=\small,
		outer sep=3pt,
	},
	edge style/.style={
		#1,
		text=black,
		font=\footnotesize,
		above
	}
}

\makeatletter
\newcommand\SoulColor{%
	\let\set@color\beamerorig@set@color
	\let\reset@color\beamerorig@reset@color}
\makeatother

\lstset{language=XML}

\title{Einführung in RDF Schema}
\author{Markus Stocker}
\date{4. Juni 2018}

\begin{document}

\maketitle

\begin{frame}{Rekapitulation}
	
	\begin{itemize}
		\item Was ist ein \emph{triple pattern}?
		\item Welche Resultatformate gibt es in SPARQL?
		\item Wozu verwendet man \texttt{LIMIT} und \texttt{OFFSET}?
		\item Was leistet SPARQL Update?
		\item Worum geht es bei der Abfragenoptimierung?
		\item In welchem Zusammenhang haben wir über Schema gesprochen?
		\item Welche Sprachen haben wir angeschaut?
	\end{itemize}
	
\end{frame}

\begin{frame}{Übersicht}
	
	\begin{itemize}
		\item Was ist RDF Schema?
		\item Konzepte und Sprachkonstrukte
		\item Beispiele
	\end{itemize}
	
\end{frame}

\begin{frame}{Schema: Schon wieder?}
	
	\begin{itemize}
		\item Tripel als fundamentale Einheit in RDF
		\item RDF Daten ergeben eine Tripelmenge
		\item Solche Mengen können sehr gross sein
		\item Milliarden, sogar Billionen oder mehr
		\item Solche Mengen sind unübersichtlich
		\item Worüber sagen die Tripel etwas aus?
		\item Eine Übersicht zu erhalten ist auf solchen Mengen schwierig
		\item Schema kommt dabei zur Hilfe
	\end{itemize}
	
\end{frame}

\begin{frame}{RDF Schema}
	
	\begin{itemize}
		\item RDF Schema ermöglicht das Organisieren von Tripelmengen
		\item Es ist eine Sprache und wird meist mit RDFS gekürzt
		\item Version 1.0 seit 2004 eine W3C Empfehlung (\emph{Recommendation})
		\item Seit 2014 in der Version 1.1 (ebenfalls W3C Empfehlung)
	\end{itemize}
	
\end{frame}

\begin{frame}{RDF Schema}
	
	\begin{itemize}
		\item Zentral ist die Gruppierung von Ressourcen
		\item Wie z.B. \texttt{(a,b,...)} sind Planeten; \texttt{(x,y,...)} sind Satelliten
		\item RDFS stellt Konstrukte zur Verfügung 
		\item Die u.A. solche Gruppierungen ermöglichen
		\item Gruppen werden \emph{Klassen} genannt
		\item Man kann mit RDFS also aussagen, dass \texttt{K} eine Klasse ist
		\item \texttt{K} steht hier für die Klasse der z.B. Planeten, Tiere, Studenten, ...
	\end{itemize}
	
\end{frame}

\begin{frame}[fragile]{Ressourcen Gruppieren: Beispiel}
	
    \begin{center}
    	\begin{tabular}{c}
			\begin{lstlisting}
ex:ghf5 rdfs:label "Moon" .
ex:12bv ex:radius "6371" .
ex:fg54 rdfs:label "Mars" .
ex:12bv ex:satellite ex:ghf5 .
ex:12bv rdfs:label "Earth" .
			\end{lstlisting}
		\end{tabular}
	\end{center}
	
	\begin{itemize}
		\item Folgende Ressourcen: \texttt{ex:12bv, ex:fg54, ex:ghf5} 
		\item Diese kann man gruppieren, und zwar
		\item \texttt{(ex:12bv, ex:fg54)} und \texttt{(ex:ghf5)}
		\item Man erhält also zwei Klassen
	\end{itemize}
	
\end{frame}

\begin{frame}{Klassen}
	
	\begin{itemize}
		\item Klassen sind benannt, mittels URIs
		\item Klassenbildung führt somit zu neuen Terme
		\item Dies sind Elemente eines Vokabulars
		\item Ein Vokabular das beschreibt worüber eine Tripelmenge ``spricht''
	\end{itemize}
	
\end{frame}

\begin{frame}[fragile]{Klassen Benennen: Beispiel}
	
    \begin{center}
    	\begin{tabular}{c}
			\begin{lstlisting}
ex:ghf5 rdfs:label "Moon" .
ex:12bv ex:radius "6371" .
ex:fg54 rdfs:label "Mars" .
ex:12bv ex:satellite ex:ghf5 .
ex:12bv rdfs:label "Earth" .
			\end{lstlisting}
		\end{tabular}
	\end{center}
	
	\begin{itemize}
		\item Die Klasse der Planeten: \texttt{ex:Planet = (ex:12bv, ex:fg54)} 
		\item Die Klasse der Satelliten: \texttt{ex:Satellite = (ex:ghf5)}
	\end{itemize}
	
\end{frame}

\begin{frame}{Klassen Spezifizieren}
	
	\begin{itemize}
		\item URI alleine genügt nicht um auszusagen, dass ein Name eine Klasse ist
		\item Sind \texttt{ex:Planet}, \texttt{ex:fg54} Klassen? Unbestimmt
		\item Man muss Klassennamen explizit als Klassen definieren
		\item Dafür stellt RDFS Konstrukte zur Verfügung
		\item Insbesondere die ``Klasse aller Klassen'': \texttt{rdfs:Class}
		\item Diese ``Metaklasse'' ist Teil der RDFS Sprache
	\end{itemize}
	
\end{frame}

\begin{frame}[fragile]{Klassen Spezifizieren: Beispiel}
	
    \begin{center}
    	\begin{tabular}{c}
			\begin{lstlisting}
ex:Planet rdf:type rdfs:Class .
			\end{lstlisting}
		\end{tabular}
	\end{center}
	
\end{frame}

\begin{frame}{Klassen Instanzen}
	
	\begin{itemize}
		\item Hat man Klassen definiert, können Instanzen erzeugt werden
		\item Instanzen sind Elemente einer Klasse
		\item Das Prädikat \texttt{rdf:type} definiert die Klasse einer Instanz
		\item So typisiert man Ressourcen
		\item Sprich, definiert Ressourcen als Elemente einer Menge
	\end{itemize}
	
\end{frame}

\lstset{escapeinside={<@}{@>}}
\begin{frame}[fragile]{Klassen Instanzen: Beispiel}
	
    \begin{center}
    	\begin{tabular}{c}
    		\begin{lstlisting}
ex:Planet rdf:type <@\textcolor{red}{rdfs:Class}@> .
ex:12bv rdf:type ex:Planet .
    		\end{lstlisting}
    	\end{tabular}
    \end{center}
    
    \vspace{0.5cm}
    
	\begin{itemize}
		\item \texttt{rdfs:Class} ist die Klasse aller Klassen
	\end{itemize}
	
\end{frame}

\lstset{escapeinside={<@}{@>}}
\begin{frame}[fragile]{Klassen Instanzen: Beispiel}
	
	\begin{center}
		\begin{tabular}{c}
			\begin{lstlisting}
<@\textcolor{red}{ex:Planet rdf:type rdfs:Class}@> .
ex:12bv rdf:type ex:Planet .
			\end{lstlisting}
		\end{tabular}
	\end{center}
	
	\vspace{0.5cm}
	
	\begin{itemize}
		\item {\color{lightgray} \texttt{rdfs:Class} ist die Klasse aller Klassen}
		\item \texttt{ex:Planet} ist eine Instanz der Klasse aller Klassen
		\item \texttt{ex:Planet} ist somit eine Klasse
	\end{itemize}
	
\end{frame}

\lstset{escapeinside={<@}{@>}}
\begin{frame}[fragile]{Klassen Instanzen: Beispiel}
	
	\begin{center}
		\begin{tabular}{c}
			\begin{lstlisting}
ex:Planet rdf:type rdfs:Class .
<@\textcolor{red}{ex:12bv rdf:type ex:Planet}@> .
			\end{lstlisting}
		\end{tabular}
	\end{center}
	
	\vspace{0.5cm}
	
	\begin{itemize}
		\item {\color{lightgray} \texttt{rdfs:Class} ist die Klasse aller Klassen}
		\item {\color{lightgray} \texttt{ex:Planet} ist eine Instanz der Klasse aller Klassen}
		\item {\color{lightgray} \texttt{ex:Planet} ist somit eine Klasse}
		\item \texttt{ex:12bv} ist eine Instanz der Klasse \texttt{ex:Planet}
	\end{itemize}
	
\end{frame}

\begin{frame}[fragile]{Die Metaklasse ist eine Klasse}
	
	\begin{center}
		\begin{tabular}{c}
			\begin{lstlisting}
rdfs:Class rdf:type rdfs:Class .
			\end{lstlisting}
		\end{tabular}
	\end{center}
	
	\vspace{0.5cm}
	
	\begin{itemize}
		\item Die Klasse aller Klassen (die Metaklasse) ist selbst eine Klasse!
	\end{itemize}
	
\end{frame}

\begin{frame}{RDF und RDFS Klassen (Auswahl)}
	
	\begin{itemize}
		\item \texttt{rdfs:Resource}, die Klasse aller Ressourcen (Klassen, Instanzen, ...)
		\item \texttt{rdfs:Class}, die Klasse aller Klassen
		\item \texttt{rdfs:Literal}, die Klasse aller Literale (Werte)
		\item \texttt{rdfs:Datatype}, die Klasse aller Datentypen (Instanzen sind Klassen)
		\item \texttt{rdf:Property}, die Klasse aller Prädikate (Relationen)
	\end{itemize}
	
\end{frame}

\begin{frame}{Unterklassen (\emph{sub classes})}
	
	\begin{itemize}
		\item Wir nun haben die Klasse \texttt{ex:Planet} definiert
		\item In unserem Sonnensystem werden Planeten in zwei Gruppen unterteilt
		\item Innere Planeten und äussere Planeten
		\item Innere Planeten sind näher an der Sonne, kleiner und steinig
		\item Äussere Planeten sind weiter entfernt, grösser und bestehen aus Gasen
		\item Innere Planeten: Merkur, Venus, Erde und Mars
		\item Äussere Planeten: Jupiter, Saturn, Uranus und Neptun
		\item Natürlich sind das alle Planeten
	\end{itemize}
	
\end{frame}

\begin{frame}[fragile]{Unterklassen}
	
	\begin{center}
		\begin{tabular}{c}
			\begin{lstlisting}
ex:Planet rdf:type rdfs:Class .
			\end{lstlisting}
		\end{tabular}
	\end{center}
	
\end{frame}

\begin{frame}[fragile]{Unterklassen}
	
	\begin{center}
		\begin{tabular}{c}
			\begin{lstlisting}
ex:Planet rdf:type rdfs:Class .

ex:InnerPlanet rdf:type rdfs:Class .
ex:OuterPlanet rdf:type rdfs:Class .
			\end{lstlisting}
		\end{tabular}
	\end{center}
	
\end{frame}

\begin{frame}[fragile]{Unterklassen}
	
	\begin{center}
		\begin{tabular}{c}
			\begin{lstlisting}
ex:Planet rdf:type rdfs:Class .
ex:InnerPlanet rdf:type rdfs:Class .
ex:OuterPlanet rdf:type rdfs:Class .

ex:12bv rdf:type ex:InnerPlanet . 
ex:fg54 rdf:type ex:InnerPlanet .
ex:rs01 rdf:type ex:OuterPlanet .
ex:3op4 rdf:type ex:OuterPlanet .
			\end{lstlisting}
		\end{tabular}
	\end{center}
	
	\begin{itemize}
		\item Gut, aber es ist nun nicht klar, dass z.B. \texttt{ex:12bv} ein Planet ist
		\item Bekannt ist nur, dass \texttt{ex:12bv} ein innerer Planet ist
	\end{itemize}
	
\end{frame}

\begin{frame}[fragile]{Unterklassen: \texttt{rdfs:subClassOf}}
	
	\begin{center}
		\begin{tabular}{c}
			\begin{lstlisting}
ex:Planet rdf:type rdfs:Class .
ex:InnerPlanet rdfs:subClassOf ex:Planet .
ex:OuterPlanet rdfs:subClassOf ex:Planet .

ex:12bv rdf:type ex:InnerPlanet . 
ex:12bv rdfs:label "Earth" .
<@\textcolor{red}{ex:12bv rdf:type ex:Planet .}@>
			\end{lstlisting}
		\end{tabular}
	\end{center}
	
	\begin{itemize}
		\item \texttt{ex:InnerPlanet} ist unterklasse der Klasse \texttt{ex:Planet}
		\item \texttt{ex:OuterPlanet} ist unterklasse der Klasse \texttt{ex:Planet}
		\item Beide sind somit Klassen
		\item Explizit bekannt ist, dass \texttt{ex:12bv} Instanz von \texttt{ex:InnerPlanet} ist
		\item Implizit ist, dass \texttt{ex:12bv} auch eine Instanz von \texttt{ex:Planet} ist
		\item Weil alle inneren Planeten auch Planeten sind
	\end{itemize}
	
\end{frame}

\begin{frame}{Unterklassen: \texttt{rdfs:subClassOf}}
	
	\begin{itemize}
		\item \texttt{rdfs:subClassOf} ist ein Prädikat
		\item Es ist somit eine Instanz der Klasse \texttt{rdf:Property}
		\item Man kann damit Klassenhierarchien bilden
		\item Ein ``super class of'' Prädikat gibt es in RDFS nicht
	\end{itemize}
	
\end{frame}

\begin{frame}{Prädikate (\emph{Property})}
	
	\begin{itemize}
		\item Prädikate sind Relationen zwischen Ressourcen
		\item In RDFS sind Prädikate ebenfalls Ressourcen
		\item Allerdings als eigene Klasse organisiert
		\item Nämlich die Klasse aller Prädikate, \texttt{rdf:Property}
		\item \texttt{rdf:Property} ist somit eine Klasse, kein Prädikat
		\item Es ist eine Instanz der Klasse \texttt{rdfs:Class}
		\item Instanzen der Klasse \texttt{rdf:Property} sind aber Prädikate
		\item Tripel Prädikate werden automatisch also solche Instanzen behandelt
	\end{itemize}
	
\end{frame}

\begin{frame}[fragile]{Prädikate: Beispiel}
	
	\begin{center}
		\begin{tabular}{c}
			\begin{lstlisting}
ex:radius rdf:type rdf:Property .

ex:12bv ex:radius "6371" .
			\end{lstlisting}
		\end{tabular}
	\end{center}
	
\end{frame}

\begin{frame}{Unterprädikate (\emph{sub properties})}
	
	\begin{itemize}
		\item RDFS erlaubt die Spezifikation von Unterprädikate
		\item Dies ermöglicht die Erstellung von Prädikathierarchien
		\item Man verwendet dazu das Prädikat \texttt{rdfs:subPropertyOf}
	\end{itemize}
	
\end{frame}

\lstset{escapeinside={<@}{@>}}
\begin{frame}[fragile]{Unterprädikate: Beispiel}
	
	\begin{center}
		\begin{tabular}{c}
			\begin{lstlisting}
ex:radius rdf:type rdf:Property .
ex:physicalProperty rdf:type rdf:Property .

ex:radius rdfs:subPropertyOf ex:physicalProperty .

ex:12bv ex:radius "6371" .
<@\textcolor{red}{ex:12bv ex:physicalProperty "6371" .}@>
			\end{lstlisting}
		\end{tabular}
	\end{center}
	
\end{frame}

\begin{frame}{Prädikatrestriktionen (\emph{property restrictions})}
	
	\begin{itemize}
		\item Ermöglichen Aussagen über die Ressourcen die ein Prädikat verbindet
		\item Insbesondere Aussagen über Klassenzugehörigkeit der Ressourcen
		\item Klassenzugehörigkeit der Subjekte und Objekte eines Prädikats
		\item Beispiel: Sind \texttt{A} und \texttt{B} verheiratet dann sind beide Personen
		\item Dazu verwendet man \texttt{rdfs:domain} und \texttt{rdfs:range}
		\item \texttt{rdfs:domain}: Klassenzugehörigkeit des Subjekts
		\item \texttt{rdfs:range}: Klassenzugehörigkeit des Objekts
		\item \texttt{rdfs:domain} und \texttt{rdfs:range} sind beides Prädikate
	\end{itemize}
	
\end{frame}

\lstset{escapeinside={<@}{@>}}
\begin{frame}[fragile]{Prädikatrestriktionen: Beispiel}
	
	\begin{center}
		\begin{tabular}{c}
			\begin{lstlisting}
ex:satellite rdf:type rdf:Property .
ex:satellite rdfs:domain ex:Planet .
ex:satellite rdfs:range ex:Satellite .

ex:12bv ex:satellite ex:ghf5 .
<@\textcolor{red}{ex:12bv rdf:type ex:Planet .}@>
<@\textcolor{red}{ex:ghf5 rdf:type ex:Satellite .}@>
			\end{lstlisting}
		\end{tabular}
	\end{center}
	
\end{frame}

\begin{frame}{Zusammenfassung}
	
	\begin{itemize}
		\item Zentral in RDFS ist das Konzept der Gruppierung
		\item Gruppierung von Ressourcen als benannte Klassen
		\item Ermöglicht die Erstellung von abstraktem Vokabular
		\item Dieses beschreibt Tripelmengen
	\end{itemize}
	
\end{frame}

\end{document}