\documentclass{beamer}
\usetheme{Boadilla}
\usecolortheme{sidebartab}
\beamertemplatenavigationsymbolsempty
\setbeamertemplate{footline}[frame number]
\usepackage{hyperref} 
\usepackage{graphicx}
\usepackage{color}
\usepackage{booktabs}
\usepackage{listings}
\usepackage{soul}
\usepackage{tikz}
\usepackage[utf8]{inputenc}
\usepackage{CJKutf8}
\usetikzlibrary{shapes.geometric}

\definecolor{gray}{rgb}{0.4,0.4,0.4}
\definecolor{darkblue}{rgb}{0.0,0.0,0.6}
\definecolor{cyan}{rgb}{0.0,0.6,0.6}

\lstset{
	basicstyle=\ttfamily,
	columns=fullflexible,
	showstringspaces=false,
	commentstyle=\color{gray}\upshape
}

\tikzset{node style/.style={
		draw=blue,
		thick,
		fill=blue!70,
		text=white,
		ellipse,
		minimum width=2cm,
		minimum height=0.75cm,
		font=\small,
		outer sep=3pt,
	},
	blank style/.style={
		draw=black,
		thick,
		fill=white,
		text=white,
		ellipse,
		minimum width=2cm,
		minimum height=0.75cm,
		font=\small,
		outer sep=3pt,
	},
	literal style/.style={
		draw=red,
		thick,
		fill=red!70,
		text=white,
		rectangle,
		minimum width=2cm,
		minimum height=0.75cm,
		font=\small,
		outer sep=3pt,
	},
	edge style/.style={
		#1,
		text=black,
		font=\footnotesize,
		above
	}
}

\makeatletter
\newcommand\SoulColor{%
	\let\set@color\beamerorig@set@color
	\let\reset@color\beamerorig@reset@color}
\makeatother

\lstset{language=XML}

\title{SPARQL: Die RDF Abfragesprache}
\author{Markus Stocker}
\date{14. Mai 2018}

\begin{document}

\maketitle

\begin{frame}{Rekapitulation}
	
	\begin{itemize}
		\item Wozu wird das Prädikat \texttt{rdf:type} verwendet?
		\item Nennen Sie einige XSD Datentypen
		\item Erläutern Sie warum diese ungleich sind
		\begin{itemize}
			\item \texttt{"14.5"}
			\item \texttt{"14.5"\^{}\^{}xsd:string}
			\item \texttt{"14.5"\^{}\^{}xsd:decimal}
		\end{itemize}
		\item Welche RDF Listen gibt es und wozu verwendet man diese?
		\item Was ist Reifizierung?
	\end{itemize}
	
\end{frame}

\begin{frame}{Übersicht}
	
	\begin{itemize}
		\item Einführung in SPARQL
		\item Konzepte
		\item Beispiele
	\end{itemize}
	
\end{frame}

\begin{frame}
	
	\huge
	\begin{center}
		Was ist SQL\\und wozu verwendet man die Sprache?
	\end{center}
	
\end{frame}

\begin{frame}{Eine Abfragesprache für RDF}
	
	\begin{itemize}
		\item Mit RDF kann man Information maschinenlesbar beschreiben
		\item Zum Beispiel ``Erde ist ein Planet mit Radius 6371 km''
		\item Nun möchte man auf solche Information zugreifen können
		\item Zum Beispiel ``Alle Planeten mit Radius grösser als 6000 km''
		\item Dazu benötigt man eine Abfragesprache
	\end{itemize}
	
\end{frame}

\begin{frame}{SPARQL}

	\begin{itemize}
		\item \textbf{SPARQL} \textbf{P}rotocol \textbf{A}nd \textbf{R}DF \textbf{Q}uery \textbf{L}anguage
		\item Ermöglicht Zugang auf in RDF beschriebene Information
		\item Erinnert etwas an die Structured Query Language (SQL)
		\item Natürlich zugeschnitten auf RDF Graphen
		\item W3C Recommendation seit 2008
		\item Neuere Version SPARQL 1.1 seit 2013
	\end{itemize}
		
\end{frame}

\begin{frame}[fragile]{SPARQL: Beispiel}

	\small
	\begin{lstlisting}
  PREFIX ex: <http://example.org#> 
  PREFIX rdf: <http://www.w3.org/1999/02/22-rdf-syntax-ns#>
  PREFIX rdfs: <http://www.w3.org/2000/01/rdf-schema#>
	
  SELECT ?label
  WHERE {
    ?planet rdf:type ex:Planet .
    ?planet rdfs:label ?label .
  }
  
  =>
  
  | label															|
  | ------------------- |
  | "Earth"													|			
  | ------------------- |
	\end{lstlisting}
	
\end{frame}

\begin{frame}{\emph{Triple Pattern}}
	
	\begin{itemize}
		\item Wie besprochen, ist das Tripel ein zentrales Konstrukt in RDF
		\item Ein RDF Dokument ist eine Tripelmenge
		\item Mit Tripelmuster (\emph{triple pattern}) kann man solche Mengen abfragen
	\end{itemize}
	
\end{frame}

\begin{frame}[fragile]{\emph{Triple Pattern}: Beispiel}
	
	Ein RDF Dokument mit einem Tripel
	\vspace{0.5cm}
	\small
	\begin{lstlisting}
  @prefix ex: <http://example.org#> .
  @prefix rdf: <http://www.w3.org/1999/02/22-rdf-syntax-ns#> .
  
  ex:Earth rdf:type ex:Planet .
	\end{lstlisting}
	\vspace{1cm}\normalsize
	Ein (passendes) \emph{triple pattern}
	\vspace{0.5cm}
	\small
	\begin{lstlisting}		
  ?planet rdf:type ex:Planet .
	\end{lstlisting}
	
\end{frame}

\begin{frame}{\emph{Triple Pattern}: Variabel}
	
	\begin{itemize}
		\item Zentral für das Tripelmuster ist die Variabel
		\item Ein \emph{triple pattern} kann eine oder mehrere Variabeln enthalten
		\item Beliebig in Subjekt-, Prädikat- oder Objektposition
		\item Variabeln beginnen mit einem Fragezeichen (\texttt{?})
		\item Variabelname kann frei gewählt werden
		\item Sollte allerdings sinnvoll sein: \texttt{?planet} besser als \texttt{?p45t}
	\end{itemize}
	
\end{frame}

\begin{frame}[fragile]{Variabel: Beispiele}
	
	\small
	\begin{lstlisting}		
  ?planet rdf:type ex:Planet .
  
  ex:Earth rdf:type ?type .
  
  ex:Earth ?predicate ?object .
  
  ?subject rdf:type ?object .
  
  ?subject ?predicate ?object .
  
  ?s ?p ?o .
	\end{lstlisting}
	
\end{frame}

\begin{frame}{Abfrage Ausführen: Tripelmuster über -menge Auswerten}
	
	\begin{itemize}
		\item In einer Abfrage wird das Muster über eine Menge ausgewertet
		\item Es werden Variabeln mit RDF Knoten oder Kanten ersetzt
		\item Diese Ersetzung ergibt das Resultat der Abfrage
	\end{itemize}
	
\end{frame}

\lstset{escapeinside={<@}{@>}}
\begin{frame}[fragile]{Muster über Menge Auswerten: Beispiel}
	
	Auswertung Tripelmuster
	\vspace{0.1cm}\small
	\begin{lstlisting}		
  ?planet rdf:type ex:Planet .
    \end{lstlisting}
   	\vspace{0.3cm}\normalsize
    über die Tripelmenge (Tripel die auf Muster passen in Blau)
   	\vspace{0.1cm}\small
	\begin{lstlisting}
  ex:Earth rdfs:label "Earth" .
  <@\textcolor{blue}{ex:Earth rdf:type ex:Planet}@> .
  ex:Mars rdfs:label "Mars" .
  ex:Earth ex:radius "6371"^^xsd:decimal .
  <@\textcolor{blue}{ex:Mars rdf:type ex:Planet}@> .
    \end{lstlisting}
   	\vspace{0.3cm}\normalsize
    Resultat der Abfrage mit zwei Variabelersetzungen
   	\vspace{0.1cm}\small
	\begin{lstlisting}
  1. ?planet <- ex:Earth
  2. ?planet <- ex:Mars
	\end{lstlisting}
\end{frame}

\begin{frame}{\emph{Basic Graph Pattern}}
	
	\begin{itemize}
		\item Mit einem \emph{triple pattern} kann man keine komplexe Abfragen stellen
		\item Dazu benötigt man eine \emph{triple pattern} Menge
		\item Auch \emph{basic graph pattern} (BGP) genannt
		\item Ein BGP kann natürlich aus einem einzigen \emph{triple pattern} bestehen
	\end{itemize}
	
\end{frame}

\begin{frame}[fragile]{\emph{Basic Graph Pattern}: Beispiel}
    
    \Large
	\begin{lstlisting}		
  ?planet rdf:type ex:Planet .
  ?planet ex:radius ?radius .
	\end{lstlisting}
	
\end{frame}

\begin{frame}[fragile]{Verbindungen (\emph{Joins})}
	
	\begin{itemize}
		\item Variabelnamen definieren Verbindungen zwischen \emph{triple patterns}
		\item Können beliebig über Subjekte, Prädikate, Objekte definiert werden
		\item Gleichbenannte Variabeln müssen mit gleichen Knoten ersetzt werden
	\end{itemize}
	
	\vspace{0.5cm}
	\large
	\begin{lstlisting}		
  <@\textcolor{red}{?planet}@> rdf:type ex:Planet .
  <@\textcolor{red}{?planet}@> ex:radius ?radius .
  <@\textcolor{red}{?planet}@> ex:satellite <@\textcolor{blue}{?satellite}@> .
  <@\textcolor{blue}{?satellite}@> rdfs:label ?label .
    \end{lstlisting}
	
\end{frame}

\begin{frame}[fragile]{\emph{Group Graph Pattern}}
	
	\begin{itemize}
		\item \emph{Basic graph patterns} können gruppiert werden
		\item Dazu verwendet man geschweifte Klammern, \texttt{\{} und \texttt{\}}
		\item Bedingungen (z.B. \texttt{FILTER}) können so auf Gruppen limitiet werden
	\end{itemize}
	
	\begin{lstlisting}
  {
    ?planet rdf:type ex:Planet .
    ?planet ex:radius ?radius .
  }

  {
    { ?planet rdf:type ex:Planet . }
    { ?planet ex:radius ?radius . }
  }
	\end{lstlisting}
	    
\end{frame}

\begin{frame}[fragile]{\texttt{FILTER}}

	\begin{itemize}
		\item \texttt{FILTER} ermöglicht Angabe von Bedingungen
		\item Variabelnersetzungen müssen diese erfüllen
		\item Somit eine Einschränkung der Resultatsmenge
		\item Geltungsbereich ist das \emph{group graph pattern} welches \texttt{FILTER} definiert
	\end{itemize}
	
	\begin{lstlisting}
  {
    ?planet rdf:type ex:Planet .
    ?planet ex:radius ?radius .
    FILTER (?radius > 6000)
  }
	\end{lstlisting}

\end{frame}

\begin{frame}{\texttt{FILTER}: Operatoren}
	
	\begin{itemize}
		\item Vergleichs, boolesche, arithmetische und spezielle Operatoren
		\item Vergleichsoperatoren: \texttt{=}, \texttt{>}, \texttt{<}, \texttt{>=}, \texttt{<=}, \texttt{!=}
		\item Definiert für Datentypen mit natürlicher Ordnung, inklusive Zeit
		\item Boolesche Operatoren: \texttt{\&\&}, \texttt{||}, \texttt{!}
		\item Arithmetische Operatoren: \texttt{+}, \texttt{-}, \texttt{*}, \texttt{/}
		\item Definiert für numerische Werte
		\item Einige spezielle Operatoren: \texttt{LANG()}, \texttt{DATATYPE()}, \texttt{REGEX()} 
	\end{itemize}
	
\end{frame}

\begin{frame}[fragile]{\texttt{FILTER} Operatoren: Beispiele}
	
	\begin{lstlisting}
  ?planet rdf:type ex:Planet .
  ?planet rdfs:label ?label .
  ?planet ex:radius ?radius .
  
  FILTER (?radius > 6000 && REGEX(?label, "^E"))
  FILTER (DATATYPE(?radius) = xsd:decimal)
  FILTER (LANG(?label) = "en")
	\end{lstlisting}
	
\end{frame}

\begin{frame}[fragile]{\texttt{OPTIONAL}}
	
	\begin{itemize}
		\item Ermöglicht die Definition von optionalen \emph{group graph patterns}
		\item Optionale \emph{patterns} müssen Variabeln nicht zwingend ersetzen
		\item Fehlt die Ersetzung wird das Resultat nicht aus der Menge genommen
		\item Variabeln in optionalen \emph{patterns} können somit ungebunden sein
	\end{itemize}
	
	\small
	\begin{lstlisting}
  {
    ?planet rdf:type ex:Planet .
    OPTIONAL { ?planet ex:radius ?radius }
  }
  
  | planet				| radius														|
  | ----------| ------------------- |
  | ex:Earth		| "6371"^^xsd:decimal |
  | ex:Mars			|																					|
  | ----------| ------------------- |
	\end{lstlisting}
	
\end{frame}

\begin{frame}[fragile]{\texttt{OPTIONAL}: Beispiele}
	
	\small
	\begin{lstlisting}
  {
    ?planet rdf:type ex:Planet .
    OPTIONAL { 
      ?planet ex:radius ?radius .
      ?planet ex:satellite ?satellite .
    }
  }
  
  {
    ?planet rdf:type ex:Planet .
    OPTIONAL { ?planet ex:radius ?radius . }
    OPTIONAL { ?planet ex:satellite ?satellite . }
  }
	\end{lstlisting}
	
\end{frame}

\begin{frame}[fragile]{\texttt{UNION}}
	
	\begin{itemize}
		\item Ermöglicht die Definition von alternativen \emph{group graph patterns}
		\item Bildet die Vereinigungsmenge der unabhängigen Resultatsmengen
	\end{itemize}
	
	\small
	\begin{lstlisting}
  {
    ?planet rdf:type ex:Planet .
    { ?planet ex:radius ?radius . }
    UNION
    { ?planet ex:satellite ?satellite . }	
  }
	\end{lstlisting}
	
\end{frame}

\begin{frame}[fragile]{Literale in SPARQL}
	
	\begin{itemize}
		\item RDF unterscheidet typisierte und untypisierte Literale
		\item Somit sind \texttt{"6371"} und \texttt{"6371"\^{}\^{}xsd:decimal} nicht gleich
		\item Datentypen in Abfragen könnnen die Resultatsmenge verändern
	\end{itemize}
	
	\small
	\begin{lstlisting}
  ?planet ex:radius "6371" .
  
  ?planet ex:radius "6371"^^xsd:decimal .
  
  ?planet ex:radius "6371.0"^^xsd:decimal .
    
  ?planet ex:radius "6371"^^xsd:string .
  
  ?planet ex:radius 6371 .
		\end{lstlisting}
	
\end{frame}

\begin{frame}{\emph{Blank Nodes}}
	
	\begin{itemize}
		\item Nicht-ausgezeichnete (\emph{non-distinguished}) Variabeln
		\item Keine Referenzen auf spezifische RDF \emph{blank nodes}
		\item Können in Form \texttt{\_:abc} oder \texttt{[]} geschrieben werden
		\item Werden in der Abfrageverarbeitung ersetzt
		\item Sind aber nicht Teil der Resultatsmenge
		\item Der gleiche \emph{blank node} kann nicht in zwei BGPs vorkommen
	\end{itemize}
	
\end{frame}

\begin{frame}[fragile]{\emph{Blank Nodes}: Beispiele}
	
	\small
	\begin{lstlisting}
  # Mit ausgezeichneten Variabeln
  ?planet rdf:type ex:Planet .
  ?planet ex:radius ?radius .
  ?planet ex:satellite ?satellite .
  ?satellite rdfs:label ?label .
     
  # Mit blank nodes (nicht-ausgezeichneten Variabeln)
  _:a1 rdf:type ex:Planet .
  _:a1 ex:radius ?radius .
  _:a1 ex:satellite _:a2 .
  _:a2 rdfs:label ?label .
  
  # Mit blank nodes (verkuerzte Form)
  [] rdf:type ex:Planet ;
     ex:radius ?radius ;
     ex:satellite [ rdfs:label ?label ]
	\end{lstlisting}
	
\end{frame}

\begin{frame}{Zusammenfassung}
	
	\begin{itemize}
		\item Um RDF Daten zu verarbeiten benötigt es einer Abfragesprache
		\item Mittels SPARQL flexibel auf RDF Daten zugreifen
		\item Indem die gewünschte Resultatsmenge deklariert wird
		\item Das \emph{triple pattern} als zentrales SPARQL Konstrukt
		\item Subjekt, Prädikat, Objekt dürften Variabel sein
		\item Eine Abfrage wertet solche Muster über Tripelmenge aus
	\end{itemize}
	
\end{frame}

\end{document}